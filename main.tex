\documentclass[12pt,a4paper]{article} % başlık sayfalı 12pt metin
% yazı girdisi ayarları
\usepackage{times} % times new roman fontu kullan
\usepackage[utf8]{inputenc} % tex dosyası utf-8 karakter desteği

% yazı çıktısı ayarları
\usepackage[margin=4cm]{geometry} % özel kenar boşlukları
\usepackage{indentfirst} % ilk paragrafı da içeriden başlat
\usepackage[none]{hyphenat} % kesme işareti kullanımını iptal et
\usepackage{setspace}\onehalfspacing % 1.5 birim satır aralığı
\sloppy % latexin sıkı kelime aralığı kurallarını serbestleştir. Hull hatalarını giderir.

% kapakta kullanılan yerleştirmeler
\usepackage[absolute,overlay]{textpos} % koordinatlı metin kutuları
\setlength{\TPHorizModule}{1mm} % metin kutusu x birimi
\setlength{\TPVertModule}{1mm} % metin kutusu y birimi

% başlıkları Türkçeleştir
\renewcommand{\contentsname}{İÇİNDEKİLER}
\renewcommand{\listfigurename}{ŞEKİL LİSTESİ}
\renewcommand{\listtablename}{TABLO LİSTESİ}
\renewcommand{\refname}{REFERANSLAR}

% işe yarar paketler
\usepackage{blindtext} % lorem ipsum 
\usepackage{graphicx} % resim ekleyebil

\begin{document}

\begin{titlepage}
% maltepe logosu
\begin{textblock}{0}(20,20)
	\includegraphics[scale=2.0]{images/maltepe}
\end{textblock}

\begin{textblock}{100}(100,50)
	\centering
	\begin{large}
	T.C.
		
	MALTEPE ÜNİVERSİTESİ
	\end{large}
\end{textblock}

% maltepe ayracı
\begin{textblock}{0}(-10,77)
	\includegraphics[width=220mm,height=1cm]{images/maltepe_hbar}
\end{textblock}

\begin{textblock}{210}(0, 110)
	\centering
	\begin{large}
	FEN BİLİMLERİ ENSTİTÜSÜ
	
	\vspace{\baselineskip}
	BİLGİSAYAR MÜHENDİSLİĞİ ANABİLİM DALI
	\end{large}
\end{textblock}

\begin{textblock}{210}(0, 145)
	\centering
	
	\begin{large}	
		\textbf{VEKTÖREL BİÇİMDEKİ EL YAZISI KARAKTERLERİNİN}
		
		\vspace{\baselineskip}
		\textbf{BENZERLİK ANALİZİ VE SINIFLANDIRILMASI}
	\end{large}
\end{textblock}

\begin{textblock}{210}(0, 180)
	\centering
	
	\begin{large}	
		\textbf{NURETTİN ONUR TUĞCU}
		
		\vspace{\baselineskip}
		Yüksek Lisans Tezi
	\end{large}
\end{textblock}

\begin{textblock}{210}(0, 215)
	\centering
	
	\begin{large}	
		\textbf{Tez Danışmanı}
		
		\vspace{\baselineskip}
		\textbf{Yrd. Doç. Dr. Turgay Tugay Bilgin}
	\end{large}
\end{textblock}
\begin{textblock}{210}(0, 250)
	\centering
	
	\begin{large}	
		\textbf{İSTANBUL – 2013}
	\end{large}
\end{textblock}
\end{titlepage}
\pagenumbering{gobble}
\section*{}
\clearpage

\section*{}
\clearpage

\begin{Large}
\centering
\textbf{T.C.}

\vspace{\baselineskip}
\textbf{MALTEPE ÜNİVERSİTESİ}

\vspace{\baselineskip}
\textbf{FEN BİLİMLERİ ENSTİTÜSÜ}

\vspace{\baselineskip}
\textbf{BİLGİSAYAR MÜHENDİSLİĞİ ANABİLİM DALI}

\vspace{\baselineskip}
\textbf{Vektörel Biçimdeki El Yazısı Karakterlerinin Benzerlik Analizi ve Sınıflandırılması}

\vspace{\baselineskip}
\textbf{YÜKSEK LİSANS TEZİ}

\vspace{\baselineskip}
\textbf{NURETTİN ONUR TUĞCU}

\vspace{\baselineskip}
\textbf{Tez Danışmanı}

\vspace{\baselineskip}
\textbf{Yrd. Doç. Dr. Turgay Tugay Bilgin}

\vspace{\baselineskip}
\textbf{İSTANBUL – 2013}

\end{Large}
\clearpage
\pagenumbering{roman}
\setlength\parindent{1cm}
% !TeX encoding = UTF-8
\section*{\centering ÖZET}
\blindtext

\blindtext
\clearpage

\section*{\centering ABSTRACT}
\addcontentsline{toc}{section}{ABSTRACT}
\blindtext

\blindtext
\clearpage

\section*{\centering TEŞEKKÜR}
\addcontentsline{toc}{section}{TEŞEKKÜR}
\blindtext
\clearpage

% !TeX encoding = UTF-8
\tableofcontents
\clearpage
\listoffigures
\clearpage
\listoftables
\clearpage
\pagenumbering{arabic}
\setlength\parindent{0cm}

\section{GİRİŞ}
\blindtext
\clearpage

\section{GELİŞME}
\blindtext
\clearpage

\section{SONUÇ}
\blindtext
\cite{Armand2006,Arora2008,Bahlmann2002}
\clearpage

\addcontentsline{toc}{section}{REFERANSLAR}
\bibliographystyle{plain}
\bibliography{referans}
\end{document}
